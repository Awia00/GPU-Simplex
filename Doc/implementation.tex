% !TeX spellcheck=en_GB
\section{Implementation}

This section contains a description of our implementation of parallel Multi-Simplex in the GPU-parallel programming language Futhark.

\subsection{Futhark Implementations}
The implementation of outer parallel Multi-Simplex can be seen in file \texttt{simplex-outer-parallel.fut}.

\newpar
The implementation of inner parallel Multi-Simplex can be seen in file \texttt{simplex-inner-parallel.fut}.

\newpar
The implementation of fully parallel Multi-Simplex can be seen in file \texttt{simplex-full-parallel.fut}.

\subsection{Test Instance Generation}
A Simplex test instance consists of the coefficient vector, the constraints matrix and the constants vector. The coefficients and the constraints are random integers between 1 and 100. The constants are random integers between 100 and 500 to make sure the instances do not converge too quickly. Additionally, we ensure that all columns in the constraints matrix contain at least one non-zero variable. This is to ensure that the linear program is not unbounded.

Afterwards, the instance is solved by an off-the-shelf algorithm to ensure that the solution can be solved and to provide the expected output. 

The instance generator can be used to generate multiple instances of different sizes to allow testing how differing dimension lengths and number of instances can influence the running time of the different implementations. Since Futhark does not allow irregular arrays, the test data for the versions that are not fully parallel must be padded with zeroes. \todo{also make sure the variable range is up to the constant value}

To ensure a fair comparison between the implementations, the flattening of the input is precomputed. This resembles the real world use case where data could simply be generated in the correct format.

\subsection{Benchmarking}
To compare to our solution with an off-the-shelf algorithm, the CPLEX framework was chosen, which is developed and maintained by IBM\footnote{\url{http://www-01.ibm.com/software/commerce/optimization/cplex-optimizer/index.html}}. It is one of the most used optimization frameworks and can solve both linear and integer linear programs efficiently.

We tested the GPU implementations on a CentOS 7 Server with an Intel E5-4660 V4 processor, 126 GB memory and with Nvidia GTX 780ti. Both the CPLEX framework and the sequentiel 
implementation were tested on a Ubuntu 14.04 machine with 8 GB memory and an Intel i5-4200U, due to the fact that we did not have permission to install CPLEX on the server.

\newpar We have chosen to benchmark on four categories of tests which represent instances where different dimensions are the dominant or limited. By doing so, different weaknesses in the implementations can be revealed. First category is one instance where the dimensions of the linear program is large. Second category is many small instances to showcase how well the algorithm handles parallelism across instances. Third category contains many instances of big size to allow for huge parallelism and the fourth category contains many instances of varying size to simulate potential real data which might not always be uniform.

\newpar To run all the test categories, all the instance sizes with all the implementations, download \todo{download} and then run \todo{command}. This requires that you run on a machine with a GPU and that Futhark and Python are installed.