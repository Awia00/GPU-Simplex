% !TeX spellcheck=en_GB
\section{Introduction}
Linear programming is one of the most used theories in optimization and can among other things be used to solve NP-hard problems when constraining the solutions to be integers only. The Simplex algorithm (hereafter Simplex) is a widely used algorithm for solving linear programs. While Simplex has worst-case exponential running time it is very fast in practise and is still used in most of the off-the-shelf linear programming frameworks. 

For a lot of problems, using a single linear program might become impractical due to the size of the input or simply because certain parts of the problem are independent. For example in the travelling salesman problem, it might make sense to partition the problem into independent linear programs for each city or country\todo{reformulate from "example" usage}. Multiple researchers and practitioners have made both CPU and GPU parallel versions of Simplex but to our knowledge none of them work on parallelism across multiple Simplex instances.

\newpar In this project, we try to create three different massively parallel implementations of the simplex algorithm on multiple instances. The implementations are written in the data-parallel, purely functional language Futhark, which compiles to highly efficient GPU code. Furthermore, we will benchmark and compare the different parallel algorithms to assess the possible speed-ups as well as compare them with CPU versions of Simplex. There will be a focus on how flattening can be used to achieve higher levels of parallelism and how well it scales with respect to the overhead created by the flattening techniques.

