% !TeX spellcheck=en_GB
\section{Introduction}
Linear programming is one of the most used theories in optimization and can be used to solve NP-hard problems when constraining the solutions to be integers. The first linear programming algorithm is the Simplex algorithm discovered by George Dantzig in 1953. While the algorithm is exponential it is very fast in practise and is still used in most of the shelf optimization algorithms. For a lot of problems using a simple linear program becomes inpractical, for example in the traveling salesman problem, it might make sense to partition the problem into independent linear programs for each city or country. Multiple researchers and practitioners have made both CPU\todo{footnote to ref} and GPU\todo{footnote to ref} parallel versions of Simplex but to our knowledge none of them work on parallelism across simplex instances.

\newpar In this project we try to create three different massively parallel implementations of the simplex algorithm on multiple instances. The implementations are written in the data-parallel, purely functional language Futhark which can compile highly efficient GPU code. Furthermore we will benchmark and compare the different algorithms to see what speed-up can be achieved when nested parallelism is removed. 
